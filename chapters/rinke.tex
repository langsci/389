\documentclass[output=paper]{langscibook}
\ChapterDOI{10.5281/zenodo.12090434}
\author{Esther Rinke\orcid{}\affiliation{Goethe University Frankfurt} and Cristina Flores\orcid{}\affiliation{Universidade do Minho} and Jacopo Torregrossa\orcid{}\affiliation{Goethe University Frankfurt}}
\title[How different types of complexity in language acquisition]
      {How different types of complexity can account for difficult structures in bilingual and monolingual language acquisition}
\abstract{Certain linguistic structures are more challenging than others for bilingual speakers. This is true across different languages and language combinations. In this paper, we propose an account in terms of different types of linguistic complexity. Our argumentation derives from the results of a study based on a cloze test including 40 different linguistic structures of European Portuguese (EP). 180 children participated, all of them acquiring EP as a heritage language in Switzerland with different environmental languages (60 French-EP, 60 (Swiss) German-EP, and 60 Italian-EP bilinguals). The results show that the structures with the lowest accuracy rates are the same across the three groups. We single out four of these structures, namely, (i) \textit{que} as a subject relative pronoun and as a consecutive conjunction, (ii) third person clitic pronouns in different forms and syntactic constellations, (iii) simple and contracted forms of prepositions, and (iv) the inflected infinitive in a concessive construction. We show that the difficulty of these structures reflects different forms of linguistic complexity: derivational complexity, memory-based learning, context dependency of rules and multiple form-function mappings. These forms of complexity cause difficulties also in monolingual acquisition.

\keywords{monolingual and bilingual language acquisition, heritage speakers, European Portuguese, cloze test, types of linguistic complexity }
}
\IfFileExists{../localcommands.tex}{
  \addbibresource{../localbibliography.bib}
  \usepackage{tabularx,multicol}
\usepackage{url}
\urlstyle{same}
\usepackage{multirow}

\usepackage{stmaryrd}
\usepackage{soul}
\usepackage{enumitem}

\usepackage{siunitx}
\sisetup{group-digits=none}

\usepackage{langsci-optional}
\usepackage{langsci-lgr}
\usepackage{langsci-textipa}
\usepackage{langsci-branding}

\usepackage{tikz-qtree}

\usepackage{pgfplots}
\usepackage{qtree}
\qtreecenterfalse
\usepackage{tree-dvips}
\usepackage{subcaption}

\let\clipbox\undefined
\usepackage{adjustbox}
\usepackage[linguistics, edges]{forest}
\usepackage{langsci-gb4e}

% ORCIDs in langsci-affiliations 
\usepackage{orcidlink}
\SetupAffiliations{orcid placement=before}
\definecolor{orcidlogocol}{cmyk}{0,0,0,1}
\RenewDocumentCommand{\LinkToORCIDinAffiliations}{ +m }
  {%
    \orcidlink{#1}\,%
  }

  \newcommand*{\orcid}[1]{}


\makeatletter
\let\thetitle\@title
\let\theauthor\@author
\makeatother

\newcommand{\togglepaper}[1][0]{
   \bibliography{../localbibliography}
   \papernote{\scriptsize\normalfont
     \theauthor.
     \titleTemp.
     To appear in:
     E. Di Tor \& Herr Rausgeberin (ed.).
     Booktitle in localcommands.tex.
     Berlin: Language Science Press. [preliminary page numbering]
   }
   \pagenumbering{roman}
   \setcounter{chapter}{#1}
   \addtocounter{chapter}{-1}
}

\newbool{bookcompile}
\booltrue{bookcompile}
\newcommand{\bookorchapter}[2]{\ifbool{bookcompile}{#1}{#2}}


\forestset{
  my nice empty nodes/.style={% modified from manual page 52
    for tree={
      calign=fixed edge angles,
      calign angle=50,
    },
    delay={
      where n children=0{
        if content={}{
          content=\strut,
          anchor=north,
        }{
          align=center
        },
      }{
        if content={}{
          shape=coordinate,
          for parent={
            for children={
              anchor=north
            }
          }
        }{}
      }
    },
  },
  my pretty nice empty nodes/.style={
    for tree={
      calign=fixed edge angles,
      calign angle=50,
      parent anchor=south,
      delay={
        where n children=0{
          if content={}{
            content=\strut,
            anchor=north,
          }{
            align=center
          },
        }{
          if content={}{
            inner sep=0pt,
            edge path={\noexpand\path [\forestoption{edge}] (!u.parent anchor) -- (.south)\forestoption{edge label};}
          }{}
        }
      }
    }
  }
}


\newcommand{\sem}[1]{\mbox{$[\![$#1$]\!]$}}
\newcommand{\type}[1]{\ensuremath{\left \langle #1 \right \rangle }}
\newcommand{\lam}{\ensuremath{\lambda}}
\renewcommand{\and}{$\wedge$ }
\newcommand{\bex}{\begin{exe}}
\newcommand{\eex}{\end{exe}}
\newcommand{\bit}{\begin{itemize}}
\newcommand{\eit}{\end{itemize}}
\newcommand{\ben}{\begin{enumerate}}
\newcommand{\een}{\end{enumerate}}

\newcommand{\gcs}[1]{\textcolor{blue}{[gcs: #1]}}
\newcommand{\ash}[1]{\textcolor{orange}{[ash: #1]}}
\newcommand{\ngn}[1]{\textcolor{purple}{[ngn: #1]}}

\newcommand{\firstrefdash}{}


\forestset{
fairly nice empty nodes/.style={
delay={where content={}
{shape=coordinate, for siblings={anchor=north}}{}},
for tree={s sep=4mm}
}
}


 
  \input{../localhyphenation} 
  \togglepaper[1]%%chapternumber
}{}

\begin{document}
\maketitle\pagebreak

\section{Introduction}\label{sec:rinke:1}

In this paper, we address the question whether and in which way the difficulties that heritage speakers (HSs) show with certain linguistic structures can be related to different types of linguistic complexity.

As a starting point, we provide the results of a study based on a cloze test focussing on a number of different structures of European Portuguese (EP) \citep{TorregrossaRinke2022}. The test was completed by 180 bilingual children in the age span between 8 to 16 years with EP as their heritage language (HL) and different environmental languages (French, German, Italian).\footnote{Throughout the paper, we use the terms (simultaneous or early) bilingual children and heritage speakers to refer to the participants in our study. By “simultaneous bilinguals”, we refer to the acquisition type, by “heritage speakers” we refer to the socio-political context of acquisition. HLs are minority languages spoken within families with a migration background. HLs are, thus, acquired in a bilingual context where another language is the official language of the society (majority\slash societal\slash environment language). Normally, as consequence of the acquisition setting, the majority language becomes the HSs' dominant language, but this is not always the case, i.e. language (im)balance is not taken as criterion to classify HSs.} The results show that certain structures are particularly difficult for the bilingual children, whereas others are unproblematic. 

Because we find a very similar \textit{hierarchy of difficulty} across the different language combination groups, we assume that the difficulties encountered by the child HSs are, in general, independent of the environmental language. The results of the abovementioned study challenge previous accounts which assign great importance to cross-linguistic influence as a factor determining deviances in bilingual production (see \citealt{VanDijk2021} for a recent meta-analysis on cross-linguistic influences in bilingual morphosyntactic acquisition of diverse language pairs).\footnote{ \textrm{Note that this is not to say that CLI does not play any role in bilingual language acquisition. The argument goes the other way around: if we show that the same structures are complex and difficult for monolinguals and bilinguals with different language combinations, it becomes rather unlikely that CLI is the (one and only) relevant factor determining the difficulties in the acquisition of these structures by bilinguals. In any case, if CLI is argued to be a determining factor in HL development, this has to be unequivocally shown. It does not suffice to point to typological differences between the two languages of a bilingual speaker.}} 

Although we know that individual children's general proficiency is dependent on age and the amount of input that they receive in their HL (in terms of “quantity of language exposure”), it is still an open question why certain structures are more difficult to stabilize than others among bilingual as well as monolingual children.  

We argue that the complexity of the target syntactic structures is crucially involved in defining the above-mentioned hierarchy of difficulty. However, it is very difficult to define what linguistic complexity actually means, because different notions and understandings of complexity exist in the literature. In order to approach our hypothesis, we will consider the four structures that caused the most difficulties for the children tested in \citegen{TorregrossaRinke2022} study when completing the cloze test in their HL. In particular, we will focus on i) \textit{que} as a relative pronoun and consecutive conjunction, ii) clitic pronouns in different forms and syntactic constellations, iii) simple and contracted forms of prepositions and iv) the inflected infinitive in concessive sentences.

In order to show that the difficulty for the bilingual children indeed lies in the complexity of the structures (and is not related to bilingualism per se or cross-linguistic influence), we will first demonstrate that the respective structures that are difficult for bilingual children are also difficult for monolingual ones. In particular, we assume that lateness of a linguistic phenomenon in monolingual acquisition indicates its complexity for the learning\slash acquisition process. Based on previous proposals about complexity in monolingual language acquisition, we argue that complexity is a multifaceted notion. Our data allow us to identify the following types of complexity:

\begin{enumerate}[label=\roman*.]
\item derivational complexity (layers of embedding, number of movement operations, instances of merge, e.g., in relative clauses)
\item irregular and lexical forms that are memory-based (and not rule-based, e.g., lexically determined selection of “verb+preposition”)
\item context dependent rules (integration of syntactic and discourse knowledge, allomorphy dependent on phonological context, e.g., clitic allomorphy depending on the phonological context, contracted forms of prepositions in combination with definite articles) 
\item multiple form-function mappings (e.g., different functions of \textit{que, por,} the use of the inflected infinitive in certain concessive clauses) 
\end{enumerate}

\section{Empirical data: A hierarchy of difficulty in heritage EP}\label{sec:rinke:2}

It is a well known fact that certain linguistic structures cause more difficulties for bilingual speakers than others, particularly in their non-dominant language (which is often, though not always, the HL). For example, bilinguals may show more problems than monolingual speakers with phenomena like gender assignment and agreement \citep{MontrulPerpiñán2008}, case marking (\citealt{Polinsky2006,Polinsky2008Heritage}), pronoun realization and omission (\citealt{TorregrossaTsimpli2019, TorregrossaTsimpli2021}), clitic allomorphs (\citealt{RinkeFlores2014}), subjunctive \citep{FloresMarques2017}, and article realization (\citealt{MontrulIonin2010}), just to mention a few. 

In order to develop a proficiency assessment instrument for EP as HL, we constructed a cloze test, presented in detail in \citet{TorregrossaRinke2022}. In general, cloze-tests are considered to be integrative assessment tools, because the participants have to access their linguistic knowledge to reconstruct the missing gap in the test (\citealt{ChungAhn2019}).

The study was conducted in Switzerland, with bilingual children with different language combinations (Portuguese-French, Por\-tu\-guese-Ger\-man and Por\-tu\-guese-Ital\-ian), as depending on the Swiss canton of residence.

\subsection{Participants}\label{sec:rinke:2.1}

The study included 180 child HSs, 60 children for each language combination. Most of the children were born in Switzerland or emigrated there early in life. All participants acquired Portuguese from birth and the environmental language as a second first or early second language. Their age ranged from 8;6 to 16 years (M: 11;7; SD: 1;10). The study was conducted in cooperation with the \textit{Camões} Institute, where all participants attended HL classes weekly. The cloze test was conducted as an untimed written task during a HL class.

Switzerland is an ideal place to conduct this type of study, because there lives a fairly large community of Portuguese-speaking migrant families. Their children acquire the heritage language, EP, in the context of three different dominant environmental languages: French\slash German\slash Italian. In addition, Switzerland has a tight network of Portuguese HL classes, offered by the Portuguese Institute for the maintenance and development of Portuguese abroad (\textit{Instituto Camões,} see \citealt{GonçalvesVinzentin2021}).

In addition to the data presented in \citet{TorregrossaRinke2022}, we collected data from 23 monolingual Portuguese children in the ages of 12-13 years (M: 12;3; SD: 0;5) for the sake of the present discussion. They completed an online version of the same cloze test.

\subsection{Test and coding methodology}\label{sec:rinke:2.2}

The cloze-test is based on a short narrative modelled after the B3 story of the Edmonton Narrative Norms Instrument (ENNI; \citealt{BongartzTorregrossa2020, SchneiderHayward2005}). The test includes 40 gaps with a variety of structures tapping into different linguistic domains: nominal morphology, verbal morphology, (contracted and non-contracted) prepositions, different types of complementizers, (clitic) pronouns in different syntactic constellations, definite and indefinite articles, and lexical knowledge. For functional words, we deleted the whole word or provided the initial letter in order to facilitate completion and restrict the number of possible answers. For content words, we provided the first half of the word (as is usually done in c-tests) for the same reasons. The results were coded according to the following four options: correct, incorrect, missing, or not expected but correct. For the analysis, we considered the correct and unexpected (but correct) answers as “correct” (1) and the incorrect and missing answers as “incorrect” (0).

Needless to say, different structures can be difficult for different reasons. In order to be able to differentiate between structure-related factors and other causes, we also collected information on the language background of the children (age of onset to the second language, quantity of input, length of attendance of HL classes, etc.). Concerning the relevance of these factors we refer the reader to \citet{TorregrossaRinke2022}. 

\subsection{Results}\label{sec:rinke:2.3}

In \tabref{tab:rinke:1} we report the overall results for monolingual and bilingual children. Since the data collection method is different (online vs. in paper form) and the monolingual children's age range is more limited, the results have to be interpreted with caution. Nonetheless, they provide us with additional evidence related to a hierarchy of difficult structures, which we argue to hold for all children, independently of their being monolingual or bilingual.


\begin{table}
\begin{tabular}{l *2{c@{~}r}}
\lsptoprule
& \multicolumn{2}{c}{Bilinguals} & \multicolumn{2}{c}{Monolinguals}\\\midrule
Overall accuracy rate & 4635/7200 & (64.4\%) & 843/920 & (91.6\%) \\
Max                   & 170/180   & (94.4\%) & 23/23 & (100.0\%)\\
Min                   & 51/180    & (28.3\%)   & 9/23 & (39.1\%)\\
\lspbottomrule
\end{tabular}
\caption{Accuracy rates of monolingual and bilingual children}
\label{tab:rinke:1}
\end{table}

Across all 40 target structures, the 180 bilingual children show an accuracy rate of 64.4\% (4635/7200; max. 170/180 (94.4\%)/min. 51/180 (28.3\%)). The accuracy rate of the monolinguals is 91.6\% (843/920; max. 23/23 (100\%)/min. 9/23 (39.1\%)). A closer look at the results reveals that some of the structures are indeed particularly challenging for the children. The following structures received the lowest accuracy rates:

\subsubsection{\textit{que} as a relative pronoun and consecutive conjunction}

The element \textit{que} has a number of different functions in EP and occurs in different types of subordinating constructions. It may serve as a complementizer introducing a complement clause \REF{ex:rinke:1a}, a relative pronoun \REF{ex:rinke:1b} or a consecutive adverbial conjunction \REF{ex:rinke:1c}.

\ea%1
    \label{ex:rinke:1}
\ea\relax [item 18]\label{ex:rinke:1a}\\
\gll Ele pensa \textbf{que} pode ir buscar um balão    para  a    sua amiga.\\
  he  thinks that can  go bring   a    balloon for     the his  friend\\
\glt `He thinks that he can bring a balloon for his friend.’

\ex\relax [item 12]\label{ex:rinke:1b}\\
\gll Mas sem querer,       o coelhinho larga     o balão, \textbf{que} voa para longe.\\
   but without wanting the rabbit    releases the balloon that flies to     {far away}\\
\glt `Without wanting it, the rabbit releases the balloon that flies away.’

\ex\relax  [item 14]\label{ex:rinke:1c}\\
\gll A cadelinha  está tão zangada \textbf{que} começa a gritar (...).\\
  the {little dog} is     so angry      that starts    to shout\\
  \glt `The little dog is so angry that she starts to shout (…).’
\z
\z

\tabref{tab:rinke:2} shows that the constructions mentioned in \REF{ex:rinke:1b} and \REF{ex:rinke:1c} received low accuracy rates in the bilinguals' cloze test, which indicates that they are difficult for the children.


\begin{table}
\begin{tabularx}{\textwidth}{Q  >{\raggedright}p{\widthof{expected}}  >{\raggedright}p{\widthof{subject relative}}  >{\centering\arraybackslash}p{\widthof{(bilingual’s)}} }
\lsptoprule
Example with gap & expected item & grammatical category & accuracy (bilingual’s)\\
\midrule
Mas sem querer, o coelhinho larga o balão, {\longrule} voa para longe. (see \ref{ex:rinke:1b}) & [que] & subject relative pronoun & 28.3\% (51/180)\\
\tablevspace
A cadelinha está tão zangada {\longrule} começa a gritar e a discutir em voz alta com o seu amigo. (see \ref{ex:rinke:1c}) & [que] & adverbial consecutive conjunction & 49.4\% (89/180)\\
\lspbottomrule
\end{tabularx}
\caption{Accuracy rates of constructions with different types of que-subordinators}
\label{tab:rinke:2}
\end{table}

\subsubsection{Third person clitic pronouns in different forms and syntactic constellations}

Clitic pronouns in EP are marked for a number of different morphological features (e.g. gender\slash number\slash case) and can occur as simple clitics \REF{ex:rinke:2a} or contracted forms (clitic allomorphs) (\ref{ex:rinke:2b}--c). Clitic allomorphs are allomorphic forms of the clitic pronouns \textit{-o(s)}\slash\textit{-a(s)} that change for phonetic reasons due to the ending of the verb form which they attach to. For instance, in examples \REF{ex:rinke:2b} and \REF{ex:rinke:2c}, the clitic \textit{-o} (singular, \textit{-os} plural) changes its form to \textit{-lo(s}) because the verb form ends with the consonant /r/.

\ea%2
\label{ex:rinke:2}
\ea\relax [item 15]\label{ex:rinke:2a}\\
\gll Assustado, este        ouve-\textbf{a} \textit{a} \textit{gritar.}\\
  scared       this-one hears-her to cry\\
  \glt `Scared, he hears her crying.’

\ex\relax [item 7]\label{ex:rinke:2b}\\
\gll O coelhinho    quer    tirá-\textbf{lo. }\\
  the {little rabbit} wants take-it\\
\glt `The little rabbit wants to take it.’

\ex\relax [item 33]\label{ex:rinke:2c}\\
\gll e     pergunta-lhe, se poderia ajudá-\textbf{los.}\\
and asks-him        if  could    help-them\\
\glt `And he asks him whether he could help them.’
    \z
\z
\tabref{tab:rinke:3} (p.\,\pageref{tab:rinke:3}) shows the accuracy rates associated with the mentioned structures.


\begin{table}[p]
\begin{tabularx}{\textwidth}{Q >{\raggedright}p{\widthof{expected}} Q >{\centering\arraybackslash}p{\widthof{(bilingual’s)}}}
\lsptoprule
Example with gap & expected item & grammatical category & accuracy (bilingual’s)\\\midrule
Assustado, este ouve-\_ a gritar. (see \ref{ex:rinke:2a}) & {[a]} & clitic pronoun (feminine, singular, accusative) & 40\% (72/180)\\\tablevspace
O coelhinho quer tirá-\_\_ (see \ref{ex:rinke:2b}) & {[lo]} & clitic pronoun (allomorph, masculine, singular, accusative) & 38.8\% (70/180)\\\tablevspace
e pergunta-lhe se poderia ajudá-{\longrule} (see \ref{ex:rinke:2c}) & {[los]} & clitic pronoun (allomorph, masculine, plural, accusative) & 44.4\% (80/180)\\
\lspbottomrule
\end{tabularx}
\caption{Accuracy rates of (clitic) pronouns in different forms and syntactic constellations}
\label{tab:rinke:3}
\end{table}

\subsubsection{Simple and contracted forms of prepositions}

Many Portuguese prepositions can occur in a contracted form with a definite determiner. Examples are the prepositions \textit{em} + \textit{a/o} (in + the fem.\slash the masc.) = \textit{na}\slash\textit{no} and \textit{por} + \textit{a/o} (for\slash through + the fem.\slash the masc.) = \textit{pela}\slash\textit{pelo} \REF{ex:rinke:3a}. Besides prepositional phrases with an adverbial contribution, the preposition \textit{por} marks the agent of a passive verb in EP, as shown in \REF{ex:rinke:3b}. Prepositions in combination with verbs can also lead to a new verb meaning, which is semantically opaque, in the sense that it does not derive compositionally from the meaning of the verb and the one of the preposition. In example \REF{ex:rinke:3c}, the combination of the verbs \textit{ir} and \textit{ter} (go + have) with the preposition \textit{com} (with) leads to the interpretation ‘go to see’.

\ea%3
\label{ex:rinke:3}
\ea\label{ex:rinke:3a}\relax[item 2]\\
\gll decidem ir  passear \textbf{pela} \textit{floresta}\\
  decide    go walking through-the forest\\
\glt `They decide to go for a walk through the forest.’

\ex\label{ex:rinke:3b}\relax [item 34]\\
\gll A  mãe       ouve com atenção     o relato    feito \textbf{por} ele\\
 the mother listens with attention the report made by him\\
\glt `The mother listens with attention to his report.’

\ex\label{ex:rinke:3c}\relax[item 35]\\
\gll vai   ter \textbf{com} o coelho vendedor, e    pergunta-lhe pelo     preço do balão.\\
  goes have with the rabbit salesman and asks-him       for-the price of-the balloon\\
\glt `He goes to see the salesman rabbit and asks him for the price of the balloon.’
\z
\z

\noindent
\tabref{tab:rinke:4} shows that prepositions are difficult for the bilingual children, in particular in contexts like \REF{ex:rinke:3a} and \REF{ex:rinke:3b}.

\begin{table}[p]
\begin{tabularx}{\textwidth}{Q >{\raggedright}p{\widthof{expected}} Q >{\centering\arraybackslash}p{\widthof{accuracy}} }
\lsptoprule
Example with gap & expected item & grammatical category & accuracy (bilingual’s)\\\midrule
decidem ir passear p{\longrule} floresta (see \ref{ex:rinke:3a}) & {[pela]} & preposition (contraction: \textit{por + a)} & 40\% (72/180)\\\tablevspace
A  mãe ouve com atenção o relato feito {\longrule} ele, (see \ref{ex:rinke:3b}) & {[por]} & preposition (passive agent) & 42.7\% (77/180)\\\tablevspace
 vai ter {\longrule} o coelho vendedor, e pergunta-lhe pelo preço do balão. (see \ref{ex:rinke:3c}) & {[com]} & preposition (in fixed verbal expression) & 53.3\% (96/180)\\
\lspbottomrule
\end{tabularx}
\caption{Accuracy rates of (simple and contracted forms of) prepositions}
\label{tab:rinke:4}
\end{table}

\subsubsection{Inflected infinitives in concessive constructions}
\begin{sloppypar}
EP possesses a special syntactic construction: the inflected infinitive. The construction is relatively frequent, especially in final clauses introduced by the preposition \textit{para} as in \REF{ex:rinke:4a}. The inflected infinitive occurs also in concessive clauses introduced by \textit{apesar de} (‘although’, as in \ref{ex:rinke:4b}).
\end{sloppypar}

\ea%4
\label{ex:rinke:4}
\ea\label{ex:rinke:4a}
\gll Os pais       foram à         livraria      para \textbf{comprarem} os livros escolares novos.\\
     the parents went   to+the {book store} to     buy+3PPl   the school books      new\\
\glt ‘The parents went to the book store to buy the new school books.’


\ex\relax [item 28]\label{ex:rinke:4b}\\
\gll Apesar de eles \textbf{pedirem} com muita educação, ...\\
  despite of they ask+3PPl          with much education, ...\\
\glt `Although they asked delicately, ...’
\z
\z

\tabref{tab:rinke:5} shows that the inflected infinitive in concessive constructions also belongs to the difficult structures, with less than 50\% accuracy.


\begin{table}
\begin{tabularx}{\textwidth}{Q >{\raggedright}p{\widthof{[pedirem]}} Q >{\centering\arraybackslash}p{\widthof{accuracy}} }
\lsptoprule
Example with gap & expected item & grammatical category & accuracy (bilingual’s)\\
\midrule
Apesar de eles pedir\_\_ com muita educação, (see 4b) & [pedirem] & inflected infinitive 3P Plural & 47.2\% (85/180)\\
\lspbottomrule
\end{tabularx}
\caption{Accuracy rates of the inflected infinitive in concessive constructions}
\label{tab:rinke:5}
\end{table}

Taking into account the results per language combination, we find that the abovementioned structures are associated with low accuracy rates across the three groups considered in this paper, as shown in \tabref{tab:rinke:6}.


\begin{table}
\small
\begin{tabularx}{\textwidth}{ p{\widthof{[pedirem]}} Q >{\centering\arraybackslash}p{\widthof{mean: 61.5\%}} >{\centering\arraybackslash}p{\widthof{Ptg./German}} >{\centering\arraybackslash}p{\widthof{mean: 65.4\%}} }
\lsptoprule
{expected item} & {grammatical category} & {Ptg./French mean: 61.5\%}  {(1476/2400)} & {Ptg./German}  {mean: 66.1\%}  {(1588/2400)} & {Ptg./Italian}  {mean: 65.4\% (1571/2400)}\\
\midrule
{}[{que}] & subject relative pronoun & 15\%\newline (9/60) & 33.3\%\newline (20/60) & 36.6\%\newline (22/60)\\\tablevspace
{}[{que}] & adverbial consecutive conjunction & 38.3\%\newline (23/60) & 60\%\newline (36/60) & 50\%\newline (30/60)\\\tablevspace
{[lo]} & clitic pronoun (allomorph, masculine, singular, accusative) & 33.3\%\newline (20/60) & 43.3\%\newline (26/60) & 40\%\newline (24/60)\\\tablevspace
{[a]} & clitic pronoun (feminine, singular, accusative) & 38.3\%\newline (23/60) & 46.6\%\newline (28/60) & 35\%\newline (21/60)\\\tablevspace
{[los]} & clitic pronoun (allomorph, masculine, plural, accusative) & 33.3\%\newline (20/60) & 51.6\%\newline (31/60) & 48.3\%\newline (29/60)\\\tablevspace
{[pela]} & preposition (contraction:  \textit{por + a)} & 42.6\%\newline (25/60) & 43.3\%\newline (26/60) & 35\%\newline (21/60)\\\tablevspace
{[por]} & preposition (passive agent) & 38.3\%\newline (23/60) & 43.3\%\newline (26/60) & 46.6\%\newline (28/60)\\\tablevspace
{[com]} & preposition (in fixed verbal expressions) & 55\%\newline (33/60) & 53.3\%\newline (32/60) & 50\%\newline (30/60)\\\tablevspace
{}[pedirem] & inflected infinitive, 3P plural & 38.3\%\newline (23/60) & 55\%\newline (33/60) & 46.6\%\newline (28/60)\\
\lspbottomrule
\end{tabularx}
\caption{Accuracy rates for the most difficult structures across the language combination groups}
\label{tab:rinke:6}
\end{table}

\tabref{tab:rinke:7} reports for each language combination group, the 12 structures with the lowest accuracy rates in the cloze-test. We highlighted in bold the structures that were common across the three language combination groups. Notably, 11 out of the 12 structures were the same for the three groups. The Portuguese-German and Portuguese-Italian children share all 12 structures, even if in a slightly different order of accuracy. The list of structures related to the Portuguese-French children included the irregular plural noun phrase \textit{balões} (`balloons'), instead of the preposition \textit{com} (`with').


\begin{table}
\small\tabcolsep=.66\tabcolsep
\begin{tabular}{l *3{l@{~}l}}
\lsptoprule
& \multicolumn{2}{c}{Ptg./French} & \multicolumn{2}{c}{Ptg./German\footnote{At first sight, the Portuguese/German group seems to show higher accuracy rates. However, the statistical analysis in \citet{TorregrossaRinke2022} clearly shows that the language combination did not have any effect on response accuracy.}} & \multicolumn{2}{c}{Ptg./Italian}\\\midrule
1 & {[\textbf{que}]\textsubscript{SUBJ\_REL}}      & (15\%)    & {[\textbf{que}]\textsubscript{SUBJ\_REL}}     & (33.33\%)          & {[\textbf{repara}]\textsubscript{PRES.3.SG.}} & (31.67\%)\\
2 & {[\textbf{lo}]}                                & (33.33\%) & {[\textbf{pela}]}                             & (43.33\%)          & {[\textbf{pela}]}                             & (35\%)\\
3 & {[\textbf{los}]}                               & (33.33\%) & {[\textbf{lo}]}                               & (43.33\%)          & {[\textbf{a}]}                                & (35\%) \\
4 & {[\textbf{que}]\_\textsubscript{CONS}}         & (38.33\%) & {[\textbf{por}]}                              & (43.33\%)          & {[\textbf{que}]\textsubscript{SUBJ\_REL}}     & (36.67\%)\\
5 & {[\textbf{a}]}                                 & (38.33\%) & {[\textbf{a}]}                                & (46.67\%)          & {[\textbf{lo}]}                               & (40\%)\\
6 & {[\textbf{pedirem}]}                           & (38.33\%) & {[\textbf{lhe}]}                              & (50\%)             & {[\textbf{lhe}]}                              & (41.67\%)\\
7 & {[\textbf{por}]}                               & (38.33\%) & {[\textbf{vem}]}                              & {(51.67\%)} & {[\textbf{vem}]}                              & (46.67\%)\\
8 & [balões]                                     & (38.33\%) & {[\textbf{los}]}                              & (51.67\%)          & {[\textbf{por}]}                              & (46.67\%)\\
9 & {[\textbf{pela}]}                              & (41.67\%) & {[\textbf{repara}]\textsubscript{PRES.3.SG.}} & (53.33\%)          & {[\textbf{pedirem}]}                          & (48.33\%)\\
10 & {[\textbf{lhe]}}                              & (41.67\%) & {[\textbf{pedirem}]}                          & (55\%)             & {[\textbf{los}]}                              & (48.33\%)\\
11 & {[\textbf{vem}]}                              & (55\%)    & [com]                                       & (55\%)             & {[\textbf{que}]\_\textsubscript{CONS}}        & (50\%)\\
12 & {[\textbf{repara}]\textsubscript{PRES.3.SG.}} & (60\%)    & {[\textbf{que}]\_\textsubscript{CONS}}        & (60\%)             & [com]                                       & (50\%)\\
\lspbottomrule
\end{tabular}
\caption{List of the 12 most difficult structures and the associated accuracy scores for each language combination group.}
\label{tab:rinke:7}
\end{table}

\tabref{tab:rinke:7} shows that the structures that are most difficult for the bilingual children in this study are very similar across the three language combination groups. The fact that the nine linguistic structures discussed here (see i--iv above) belong to the 12 most difficult structures independently from the contact language suggests that the bilingual children’s difficulties with these structures are unlikely due to cross-linguistic influence. If it is true that these structures are associated with a complex learning task, they should be difficult for monolingual children, too.

As mentioned above, we collected data from 23 monolingually raised children, living in Portugal. The data are not fully comparable, because the cloze test was conducted online (during the COVID-19 restriction period) and included only 12-13 years-old children. As expected, the rate of accuracy was much higher in this group. Nonetheless, even in this case, we identified some difficult structures that did not reach ceiling performance. \tabref{tab:rinke:8} shows the three structures with the lowest accuracy rates, which overlap with the structures listed in \tabref{tab:rinke:7}.


\begin{table}
\begin{tabular}{rlc}
\lsptoprule
& \multicolumn{2}{c}{Portuguese monolinguals}\\
\midrule
1 & {[que]\textsubscript{SUBJ\_REL}} & (39.1\%)\\
2 & {[pela]}                         & (56.5\%)\\
3 & {[a]}                            & (78.3\%)\\
\lspbottomrule
\end{tabular}
\caption{Three most difficult structures for 12/13-years old monolinguals}
\label{tab:rinke:8}
\end{table}

These data suggest that the structures that are most difficult for bilingual children are also challenging in monolingual acquisition. However, since the data collection method is different and the children's age range is much more limited, we will support this hypothesis by also relying on existing studies on the L1 acquisition of the phenomena discussed in the next section. 

\section{A look at monolingual acquisition}\label{sec:rinke:3}

In this section, we look briefly at the main findings reported in previous literature on L1 acquisition, in Portuguese (and in other languages), of the target structures, namely at the acquisition of conjunctions, clitic pronouns, prepositions, and concessive connectors requiring the inflected infinitive.

\subsection{Subordinators and subordinate clauses}\label{sec:rinke:3.1}

Amid the various complementizers introducing subordinate clauses, \textit{que} seems to be one of the first to appear in child EP, in complement clauses (see \ref{ex:rinke:1a}), following the emergence of complement clauses with infinitives \citep{Santos2017}. However, as already mentioned in \sectref{sec:rinke:2}, \textit{que} introduces different types of subordinate clauses and we know that not all of them are acquired at the same time in EP, as in many other languages (\citealt{Soares1998}). 

Research on the acquisition of EP, in particular the study conducted by \citet{Soares1998}, has shown that relative clauses are amongst the latest types of subordinate clauses to appear in child speech (see also \citealt{Vasconcelos1995}). This has been shown also for other languages. For instance, \citet{BloomHafitz1980} and \citet{DromiBerman1986} proposed that, in English and Hebrew complement clauses emerge first, followed by adverbial clauses, and lastly, relative clauses (but see \citealt{Penner1995} for a different order in Swiss German). Various explanations have been proposed to account for the order of acquisition of different subordinate clauses \citep{Bowerman1979}. Traditionally, it is attributed to different degrees of embedding: The structure that has fewer layers of embedding is less complex and, therefore, easier to acquire. This would be the case of complement clauses, which are selected by the matrix verb in the same fashion as any other verbal complement. Adverbial clauses are not selected directly by the verb, but they involve one layer of embedding. Thus, they emerge later than complement clauses, but earlier than relative clauses, which involve both embedding and movement. Since relative clauses are the most complex structures in terms of embedding, they would be the last structure to emerge. In fact, EP children have difficulties in producing and comprehending relative clauses until school-age \citep{Vasconcelos1995}. 

\citet{Armon-Lotem2005} argues that it is necessary to look not only at the timing of emergence of certain structures, but at the timing of its complete stabilization, since a structure is only completely stabilized in the child's grammar when all the associated features are acquired and the structure is used in all relevant contexts. This explains cross-language differences and further distinctions within each type of subordination considered above. For instance, in EP there are different timings of acquisition of complement clauses due to different timings of acquisition of verbal semantics and verbal mood (\citealt{JesusSantos2019}). For relative clauses, it has been shown that right-embedded clauses emerge earlier than middle-embedded ones \citep{Vasconcelos1995} and that subject relative clauses are easier to acquire and process than object relative clauses (\citealt{CostaSilva2011}).

Furthermore, a typical property associated with the acquisition of subordination is the omission of the complementizers, which starts at a pre-conjunctional period, but is prolonged throughout the acquisition process until later stages of acquisition (\citealt{Armon-Lotem2005, Soares1998}).

For the purpose of our discussion, the main observation to retain is that, in child EP, complement clauses stabilize earlier than adverbial clauses and these stabilize earlier than relative clauses. A frequent feature of child subordination is the omission of the complementizer.

\subsection{Clitic pronouns in different forms and syntactic constellations}\label{sec:rinke:3.2}

It is a well-established fact that EP has a rich pronominal system. In addition to allowing for the use of strong and clitic pronouns, EP is also a null object language. This means that children acquiring EP have to acquire the conditions of use of strong pronouns, clitics or clitic omission, including null objects, VP ellipsis or other types of object omissions. Several studies focusing on the production and comprehension of clitics and null objects by monolingual EP children demonstrate that they go through a prolonged stage of object omission and stabilize knowledge of the pronominal system very late (at school-age; see \citealt{CostaLobo2007,CostaLobo2009, CostaSilvaEtAl2009, CostaEtAl2012, FloresSopata2020}, among others). It is argued that the overuse of null objects is caused by children’s difficulties in assigning the correct interpretation to different types of object omissions available in the target grammar (pro, variable, VP-ellipsis, null object; cf. \citealt{CostaEtAl2012}). Due to the complexity of the pronominal system, EP L1 children omit objects to a higher degree and for a longer period of time than children acquiring other Romance languages that have clitics, or even other null object languages \citep{Varlokosta2016}. Despite this delay, EP children show early pragmatic knowledge of pronoun use (\citealt{CostaSilvaEtAl2009, FloresSopata2020}). This indicates that the prolonged non-adult-like interpretation and production of pronouns lies, on the one hand, in the acquisition of the feature composition of the null objects and, on the other hand, in the acquisition of some syntactic and morphological features of clitics.

A syntactic property of clitics that has been shown to stabilize late in L1 EP is clitic placement. Differently from other clitic languages (and even differently from Brazilian Portuguese), several syntactic constraints determine the pre- or postverbal position of the clitic pronoun in EP. In particular, the preverbal position (proclisis) is stabilized very late in L1 acquisition (by age 7, see \citealt{CostaLobo2015}).

In addition to the late acquisition of the properties constraining the realization vs. omission of the object pronoun and its placement, certain morphological features also stabilize only at school-age. A case in point is mesoclisis, i.e. the occurrence of allomorphic clitic forms in the middle of the verb form (e.g. 1P Sing. future form \textit{eu vê-}\textbf{\textit{lo}}\textit{{}-ei} ‘I will see it\slash him’) and clitic allomorphy in postverbal (enclitic) position (\textit{tirá-}\textbf{\textit{lo}} ‘take it’, \textit{ajudá-}\textbf{\textit{los}} ‘help them’; see 2b and 2c). The target-like use of these structures is sensitive to formal instruction and shows variation in colloquial Portuguese (see \citealt{Catalão2011, Santos2002} and \citealt{Batalha2018} for an analysis of Portuguese school-aged children’s knowledge of clitic pronouns).

\subsection{Prepositions}\label{sec:rinke:3.3}

Prepositions are a heterogeneous category that includes elements with lexical meaning (e.g. spatial prepositions) and semantically vacuous elements functioning as grammatical markers (e.g. the dative preposition \textit{a}). Lexical prepositions have their own lexical entry, whereas non-lexical prepositions have undergone some form of grammaticalization and have a purely syntactic function or they occur in fixed phrases (\citealt{Rauh1993, Riemsdijk1990}). This split into lexical vs. functional prepositions (or a continuum from more lexical to more functional prepositions) is mirrored in the process of acquisition of languages with a prepositional system. For example, \citet{Littlefield2009} argues that in L1 English, lexical prepositions emerge early and show a steady, relatively rapid increase in child speech over time. Inversely, pure functional prepositions (e.g., `of') emerge later and their production is limited and often not target-like in the first stages of acquisition. The same seems to hold for Portuguese, even though research on the acquisition of prepositions in Portuguese is scarce \citep{Teodoro2020}.

A further characteristic of prepositions which is visible across several languages is the contraction of the preposition with other elements, such as pronouns or articles. In Portuguese, the contraction of the preposition with the definite article (see \sectref{sec:rinke:2.3}) is almost categorical, with only a few syntactic contexts representing an exception. In addition to always requiring gender and number marking, there are contractions that change the stem (e.g. \textit{por + a = pela} ‘through-the’) and contractions that involve only the deletion of the final vowel (e.g. \textit{de + a = da} ‘of-the’). Due to the absence of research on the acquisition of prepositional contractions in L1 acquisition of Portuguese, we will resort to studies on L2/L3 research (\citealt{Brito2018, PicoralCarvalho2020}). In a study with Spanish and English native\slash heritage speakers learning Portuguese as L3, \citet{PicoralCarvalho2020} show that speakers are more likely to realize contractions with the preposition \textit{a} and that the contraction of the preposition \textit{por} + definite article is the most difficult to acquire. Furthermore, the acquisition path seems to be independent of the speakers' L1.

As for the preposition \textit{por}, in addition to a spatial meaning, it has also the pure grammatical function of introducing the agent in passive sentences (as \textit{by}-phrase), either in contracted form or not, depending on the presence or absence of a definite article, respectively. It has been argued that \textit{by}{}-phrases of passive sentences are generally problematic for children (\citealt{FoxGrodzinsky1998}). This difficulty may be due to several factors, including the type of passive sentence (e.g., long or short-actional passives; see \citealt{ArmonLotemLely2016}), the agentivity of the predicate \citep{Estrela2015} and the above-mentioned difficulty for children to use semantically vacuous prepositions. 

\subsection{Inflected infinitives in concessive constructions}\label{sec:rinke:3.4}

We know from studies on spontaneous child speech that inflected infinitives emerge early in EP \citep{Santos2017}, i.e., by the age of two years. However, at an initial phase, they only occur in final clauses introduced by \textit{para} \citep{Santos2013}. Only later (i.e., by the age of three years), they occur in complements of perception verbs \citep{SantosHyams2016}. This means that even though the inflected infinitive is available to EP children from early on, the different contexts that allow its use are acquired gradually, which depends on both syntactic and lexical constraints. In fact, some contexts requiring the use of an inflected infinitive are acquired very late, i.e. in school age. This is the case for the concessive structure \textit{apesar de} (`although'). 

According to \citet{Costa2006}, the concessive connector \textit{apesar de} is stabilized very late in EP (i.e. only by the age of ten years, similar to the stabilization of \textit{although} or \textit{whereas} in English, see \citealt{Diessel2004}). \citet{Costa2006} argues that this late acquisition is caused by three different, but interacting factors. The first factor is frequency: The connectors \textit{apesar de} and \textit{embora} are produced significantly less by adults than the adversative connector \textit{mas.} However, frequency per se does not explain the late acquisition of this structure. The late stabilization of concessive connectors may be related to the fact that they occur only in subordinate clauses and most of them require the use of the subjunctive, which is also stabilized late in EP. 

\pagebreak
\section{The role of linguistic complexity}\label{sec:rinke:4}

The discussion in \sectref{sec:rinke:3} has shown that the different structures under consideration are not only difficult for bilingual children, but are also mastered relatively late by monolinguals. If these structures take time to be acquired in monolingual language acquisition, we expect to find an effect of age in the bilingual group as well. Thus, we ran a statistical analysis to assess the effect of the bilingual children's age on the acquisition of the most difficult structures. We considered the nine structures which are relevant for the present paper (see i--iv in \sectref{sec:rinke:2}). As we mentioned in \sectref{sec:rinke:1}, the age range of the participants is relatively large (i.e., from 8;6 to 16 years; \textit{M}: 11;7; SD: 1;10). We ran a binomial logistic regression with accuracy as dependent variable (0 = inaccurate, 1 = accurate) and age as fixed effect. The model showed a significant effect of age ($\beta = 0.64$, $\text{SE} = 0.06$, $z = 10.30$, $p < 0.001$). This shows that bilingual children’s knowledge of difficult structures improves with age. In this sense, bilinguals behave just like monolinguals, even if they may need more time to acquire difficult structures. In this sense, it is possible that the structures that are not mastered by younger bilingual children are exactly the structures that emerge late in monolingual language acquisition. In other words, these structures are ‘complex’ for bilinguals and monolinguals alike, as shown by their late timing of acquisition across the board. Since it is often observed that bilinguals show a more protracted development, i.e., they acquire some structures in later age spans than monolinguals, we assume that bilinguals just need some more time to catch up with their monolingual peers (see \citealt{SchulzGrimm2019, Tsimpli2014} for similar considerations). In the remainder of this paper, we intend to discuss why certain structures are associated with a more complex learning task than others.

\subsection{Notions of linguistic complexity}\label{sec:rinke:4.1}\largerpage

In the literature, complexity in acquisition has been explicitly defined and implicitly assumed in many different ways. From a syntactic perspective, it has been assumed that children initially prefer more syntactically economical structures over less economical ones; i.e. structures involving less layers of embedding over structures involving more layers of embedding, or structures involving less movement operations over structures involving more movement operations (\citealt{Hamann2006,Rizzi1990,Rizzi2000}). \citet{Jakubowicz2003} proposes that computational complexity affects child language development, leading children to produce less complex structures in a target-like way earlier than more complex structures (see also \citealt{JakubowiczNash2001}). The author develops the following \textit{Derivational Complexity Metric}.

\ea%5
\label{ex:rinke:5}Derivational Complexity Metric (DCM, \citealt{Jakubowicz2011})
\begin{itemize}
\item Merging $\alpha_i$ $n$ times gives rise to a less complex derivation than merging $\alpha_i$ $(n + 1)$ times.
\item Internal Merge of α gives rise to a less complex derivation than Internal Merge of $\alpha + \beta$.
\end{itemize}
\z

For example, with respect to wh-questions, the DCM predicts, “that the child is sensitive to the number of times that a copy of the wh-element must be merged to satisfy a computational requirement and to the number of constituents that may (or must) undergo Internal Merge (here under: IM)” (\citealt{Jakubowicz2011}: 340; see also \citealt{Soares2003} with respect to the acquisition of wh-questions in EP).

The notion of complexity presented so far is motivated syntactically. Another way of defining complexity is more morphologically oriented and based on the observation that children tend to overregularize morphological endings. \citet{ClahsenRoca2002} argue for a dual-mechanism model between rule-based (regular) and memory-based (irregular) representations for morphologically complex words. In their study, children acquiring Spanish verb morphology overapply regular paradigms to verbs that require irregular forms but not vice versa. The authors argue that “... the onset of overregularizations is syntactically triggered, by the requirement to generate a fully specified finite verb form in every sentence, in conjunction with lexical gaps or retrieval failures for irregulars. Overregularizations gradually decrease over time when children get older and memory traces for irregulars are becoming stronger and the children’s ability to retrieve them is becoming more reliable” (\citealt{ClahsenRoca2002}: 618). Coming back to the issue of complexity in acquisition, these results suggest that regular syntactic or morphological rules are less complex than irregular forms, which have to be memorized and stored in the lexicon based on individual forms (and their frequency) in the input. Hence, the acquisition of rules that are applied regularly seems to be less costly than memory-based lexical learning.

The morphological rule mentioned in the previous example is based on a syntactic requirement (namely to generate a fully specified finite verb form) that applies independently of the context (i.e., the situation in which the sentence is uttered) and, in principle, concerns every sentence. However, this is not the case for each morphological or syntactic rule. We would like to add another type of complexity which lies in-between rule-based regular and memory-based irregular representations, namely cases in which a rule is applied depending on a specific (discourse or phonological) context. We suggest that this also involves a two-step learning\slash acquisition process: acquiring the rule and understanding in which context it applies and in which context it does not. 

One example that has been mentioned in a number of studies is context dependency of a form which is related to previous discourse. In languages with null and overt pronouns, this concerns, for example, the decision whether a pronoun has to be overtly realized or can remain phonologically null. It has been suggested that in null subject languages, bilingual speakers tend to overrealize pronouns compared to monolingual speakers and may fail to accurately differentiate between the two forms in interpretation tasks. \citet[464]{SoraceBaldo2009} argue that this is a result of the complexity of the task: bilinguals have more difficulties in integrating different sources of information. According to \citet{Sorace2011}, the differences between monolingual and bilingual populations relate to bilingualism per se and, in particular, to the allocation of general cognitive resources to bilingual processing. However, the complexity of integrating syntactic information and discourse information represents a complex learning task also for young monolinguals (as shown for Portuguese by \citealt{LoboSilva2016, RinkeFlores2018}) and may result in a protracted development of such phenomena. For example, \citet{TullerBarthez2011} observe that in French, 3\textsuperscript{rd} person accusative clitics are difficult among young TD (=typically developing) children and AD (atypically developing) speakers after childhood. The authors claim that the 

\begin{quote}
complexity of object clitics is the result of a combination of several properties, the first of which is their non-canonical position. […] Summarizing, the production of accusative clitics includes the following properties: movement to a non-argument position, clustering with nominative clitics, and reference to a non-local antecedent. Production of a third person accusative clitic involves the following additional properties: establishing non-discourse-dependent reference, agreement in both number and gender, but not animacy, and, potentially, licensing of a null clitic (conditional on both lexical and discourse restrictions). They are thus complex (morpho)syn\-tac\-tic\-ally, in terms of movement (whichever analysis of clitic constructions is adopted) and agreement, and mastering their usage (knowing whether they can be null or not) requires adhering to lexical idiosyncrasies and discourse\slash pragmatic conditions.\hfill(\citealt{TullerBarthez2011}: 427f.)\hbox{}
\end{quote}

A similar observation applies to 3\textsuperscript{rd} person object clitics in EP, whose production is associated with the same degree of complexity as ascribed by \citet{TullerBarthez2011} to French clitics. In addition, EP allows for 3\textsuperscript{rd} person null objects in similar syntactic and discourse contexts as clitics. Therefore, the acquisition of the target-like distribution of object clitics and null objects in a null object language like EP and, hence, of the discourse-appropriate production of clitics is more challenging than the acquisition of clitics in non-null object languages (\citealt{CostaLobo2007, FloresSopata2020, Varlokosta2016}). 

In addition, EP clitics show allomorphy in certain phonological contexts, as described above. Allomorphic variation represents another form of linguistic complexity. It has been shown, for example, that allomorphic variation of English past tense forms (e.g. “-t for verbs such as \textit{chase}, -d for forms such as \textit{crave} and /əd/ for verbs such as \textit{recite}”) slows down morphological development (\citealt{O’GradyRees-Miller2010}: 369). \citet{O’GradyRees-Miller2010} also mention homophony as a factor determining morphological development in first language acquisition.

\begin{quote}
Whereas the word \textit{the} functions only as a determiner in English, the suffix -s can be used to mark any one of three things: plural number in nouns, third person singular in verbs, or possession. The resulting complication in the relationship between form and meaning may impede acquisition.\\\hbox{}\hfill\hbox{(\citealt{O’GradyRees-Miller2010}: 369)}
\end{quote}

We assume that in general, multiple form-function mappings (e.g. allomorphy, homophony) give rise to complexity in acquisition and may cause difficulties or a slow down in development. To conclude, we identified the following types of linguistic complexity in first language acquisition: i) derivational complexity (layers of embedding, number of movement operations, instances of merge); ii) irregular and lexical forms that are memory-based (and not rule-based); iii) context dependent rules (integration of syntactic and discourse knowledge or allomorphy depending on phonological context) and iv) multiple form-function mappings. In the next section, we will discuss how these notions of complexity apply to the “hierarchy of difficulty” discussed in \sectref{sec:rinke:2}. 

\subsection{Towards an explanation of the hierarchy of difficulty}\label{sec:rinke:4.2}

In this section, we would like to come back to the phenomena mentioned in \S 2 and \S 3 that were the most challenging linguistic structures for the children and explore to what extent their difficulty can be related to the above mentioned notions of linguistic complexity.

As already discussed in \S \ref{sec:rinke:2}, the item \textit{que} as a relative pronoun and as a consecutive complementizer belonged to the constructions with the lowest accuracy rates across the different language combination groups. It is interesting to contrast these two structures with the declarative complementizer \textit{que} illustrated in \REF{ex:rinke:1a}, which is associated with a high accuracy rate of 70.5\% (vs. 28.3\% for the relative pronoun and 49.4\% for the consecutive complementizer). As shown in \S \ref{sec:rinke:3}, the different accuracy rates for the different types of \textit{que} correspond to the order of acquisition of the different instantiations of \textit{que} in monolingual EP: the declarative complementizer is acquired first in child EP, followed by \textit{que} introducing adverbial clauses, followed in turn by relative clauses, some of which may also emerge at school age. Even the 12–13 years-old monolingual children showed low rates of accuracy in association with the relative pronoun \textit{que.} In \S \ref{sec:rinke:3}, we mentioned that the difference between the various types of subordinate clauses (complement clauses selected by the verb, adjoined adverbial clauses and relative clauses) can be accounted for in terms of degrees of derivational complexity, involving, for example, embedding (in concessive clauses) or embedding and movement (e.g., in relative clauses). An additional factor contributing to the complexity of the structures at stake is the multiple form-function mapping of \textit{que} in EP (one form with several functions), namely the homophony of \textit{que} as a conjunction of complement and adverbial clauses, as an interrogative or a relative pronoun or an interrogative determiner.

Third person clitic pronouns represent another area of difficulty in the cloze tests among the bilingual children. As mentioned in \S \ref{sec:rinke:3.2}, these structures are also very challenging for EP monolingual children and acquired successfully only at school age. Clitics are difficult for a number of different reasons. In addition to potential (syntactic) derivational complexity (if we assume a movement analysis for clitics), clitics are morphologically complex because they involve allomorphy in EP. Depending on the phonological context, the form of the clitic may change. For example, following the –r ending of infinitives, the clitic \textit{-o} (acc. masc. sing.) is realized as \textit{–lo} (see example 2b, c); after the nasal \textit{–m} (e.g., 3\textsuperscript{rd} person plural finite verb forms), \textit{o} surfaces as \textit{–no}. As discussed in the previous section, such rules are complex for different reasons: they are context dependent (therefore involving a two step learning process) and there is no direct form-function mapping (because different forms have the same function and realize the same morphological features). A third factor contributing to the complexity of clitic pronouns is their discourse dependency, since the appropriate use of clitics (as well as null objects and full noun phrases) is dependent on discourse constraints \citep{FloresSopata2020}. 

The third phenomenon discussed in \S \ref{sec:rinke:2} and \S \ref{sec:rinke:3} are prepositions in different shapes and constellations. We saw that in the cloze test, bilingual children, but also monolinguals, show low accuracy with contracted forms of prepositions. In addition, the bilingual children have also problems with the preposition \textit{por} introducing passive agents and the lexically selected preposition \textit{com}. First of all, contracted forms of prepositions are derived based on a context-dependent rule (only in combination with definite articles, not with indefinite ones or bare nouns). Assuming a Distributed Morphology approach, \citet{Ximenes2004} states that contractions of prepositions are the results of a two-step morphological process: “two operations happening in the morphological component: merger followed by fusion.” \citep[182]{Ximenes2004}. As already discussed in \S \ref{sec:rinke:3}, \textit{por} as a preposition marking the agent of a passive sentence is generally problematic for younger children and complex, because it is a functional and, hence, a semantically vacuous preposition. The homophony with the lexical preposition \textit{por} marking a spatial meaning leads to multiple form-function mapping and may contribute to the complexity of this preposition as well. The complexity of the preposition \textit{com} in combination with the verb \textit{ter} (see example \ref{ex:rinke:3c}) has a different source. In this context, the preposition contributes to the formation of a new verb – a process that is very productive in EP (e.g. \textit{acabar de} ‘finish (of)’, \textit{acabar com} ‘destroy’, \textit{acabar por} ‘end up by’). Crucially, the combination of a verb and a preposition is semantically opaque and can only be acquired through a memory-based lexical learning process.

The fourth phenomenon that was associated with some difficulty for the bilingual children in the cloze test was the inflected infinitive in combination with the concessive connector \textit{apesar de}. As mentioned in \S \ref{sec:rinke:3}, EP monolingual children do not exhibit any difficulty in the use of inflected infinitives. However, concessive connectors are acquired late and exhibit a similar degree of complexity as other conjunctions introducing adverbial phrases. When introducing a clause, \textit{apesar de} occurs only in combination with inflected and uninflected infinitives. In more formal registers, we find the (more complex) construction \textit{apesar de que,} which introduces finite subordinate clauses that require the indicative or the subjunctive mood, which is another property of EP which is acquired relatively late in L1 acquisition. In addition to these forms that belong to the standard register, we find also the occurrence of \textit{apesar que} in association with the indicative mood in colloquial speech. Furthermore, \textit{apesar de} may also introduce a NP with concessive meaning, instead of a clause (e.g., \textit{Apesar da chuva, eles foram passear.} `Despite the rain they went for a walk.'). Hence, the difficulty related to the use of the inflected infinitive in the cloze test does not depend on the structure itself, but results from its combination with the concessive connector \textit{apesar de}, which is acquired late and can introduce different structures. In particular, its alternation with a finite verb in the same context, as in the use of the indicative with the non-standard \textit{apesar que}, may increase the difficulty of the acquisition task. Actually, the most frequent error committed with this item was the replacement of the inflected infinitive with the finite 3\textsuperscript{rd} person plural indicative form \textit{pediram}.

\pagebreak
\section{Summary and conclusions}\label{sec:rinke:5}

The present line of argumentation derives from the observation that some linguistic structures cause more difficulties for bilingual speakers than others, especially in their (non-dominant) HL. We aimed to show that the difficulty of certain structures is related to different types of linguistic complexity. A cloze test conducted with 180 EP heritage children in Switzerland – divided into three groups with 3 different environmental languages (French, German and Italian) – revealed that the children exhibited particular difficulties with some structures, including relative pronouns and consecutive conjunctions, clitic pronouns in different forms and syntactic constellations, some simple and contracted forms of propositions and inflected infinitives in concessive constructions. 

Triangulating these findings with the existing literature on the L1 acquisition of the structures at issue, we were able to conclude that the structures that child HSs found the most difficult were exactly the structures that usually emerge late (or very late) in monolingual language acquisition. This was also confirmed by a small scale study conducted on Portuguese monolingual children ranging between 12 and 13 years, based on the same cloze-test as the one administered to the bilinguals. Also for the monolinguals, relative pronouns, contracted prepositions and clitics were associated with the lowest accuracy scores. 

Overall, these results suggest that the challenging structures for bilingual children represent a complex learning task also in monolingual language acquisition. In other words, child HSs acquire morphosyntactic structures through the same milestones as their monolingual peers, although they may lag behind in some linguistic domains that require more input to be successfully acquired. Notably, we also found that the accuracy in the use of these structures improved with age, highlighting a developmental trend among the bilinguals. In addition, these results do not sustain the assumption that CLI is the main factor contributing to developmental differences between heritage and monolingual children.

The main contribution of the present paper consists in showing that the difficulties exhibited by the bilingual children cannot be accounted for in terms of a single notion of complexity. Rather, different structures may be complex in acquisition for different reasons. In particular, we identified four main notions of complexity, as related to the different structures analysed in this contribution, i.e., derivational complexity, memory-based lexical forms, rules dependent on phonological or discourse contexts and multiple form-function mappings. In this sense, we moved away from the attempt to provide a single definition of complexity, but rather proposed a multifaceted view of this notion, which matches with extensive research on language acquisition.

\printbibliography[heading=subbibliography,notkeyword=this]
\end{document}
