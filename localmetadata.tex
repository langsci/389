\title{Formal approaches to complexity in heritage languages}

\BackBody{This collective volume breaks new ground in studies of linguistic complexity by addressing this phenomenon in heritage languages. It dismisses with the conception that heritage languages are less complex than their baseline or homeland counterparts and shows complexity trade-offs at various levels of linguistic representation. The authors consider defining properties of complexity as a phenomenon, diagnostics of complexity, and the ways complexity is modeled, measured, or operationalized in language sciences. The chapters showcase several bilingual dyads and offer new empirical data on heritage language production and use.}
\author{Maria Polinsky and Michael T. Putnam}
 

\renewcommand{\lsISBNdigital}{978-3-96110-478-9}
\renewcommand{\lsISBNhardcover}{978-3-98554-107-2}
\BookDOI{10.5281/zenodo.12073160}
% \typesetter{}
\proofreader{Alexandr Rosen,
Amir Ghorbanpour,
Carrie Dyck,
Cornelius Gulere,
Eliane Lorenz,
Elliott Pearl,
George Walkden,
Jean Nitzke,
Jeroen van de Weijer,
Justin Richard Leung,
Kate Bellamy,
Katja Politt,
Ksenia Shagal,
Lea Coy,
Mary Ann Walter,
Rebecca Madlener,
Vadim Kimmelman}
\lsCoverTitleSizes{47pt}{15.5mm}% Font setting for the title page

\renewcommand{\lsID}{389}

\renewcommand{\lsSeries}{cib} 
\renewcommand{\lsSeriesNumber}{3}
